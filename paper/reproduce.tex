\documentclass[10pt]{article}
\usepackage[empty]{fullpage}
\usepackage{mathpazo}
\usepackage{microtype}
\usepackage{paralist}
%\usepackage{hyperref}

\begin{document}

%%%%%%%%%%%%%%%%%%%%%%%%%%%%%%%%%%%%%%%%%%%%%%%%%%%%%%%%%%%%%%%%%%%%%%%%%%%%%%%

\begin{center}
  \Large\bfseries
  Reproducibility: Towards using the POWDER platform for RF propagation validation 
\end{center}

We have packaged our work into a POWDER profile that enables others to
replicate our results and to serve as a starting point for related research
efforts. POWDER profiles are essentially parameterized
``recipes'' that specifies the hardware and software resources required
for the experiment in question. This ``recipe'' is used by the POWDER
control framework to allocate and configure the resources in question. 
POWDER profiles also contain detailed descriptions of what to
do once the profile has been instantiated.
In the case of the RF propagation validation profile, a number of options
are possible:
\begin{inparaenum}[(1)]
\item Instantiate the profile with only compute resources and use the provided
SPLAT! software and POWDER related configuration information to perform 
RF propagation predictions.
\item Instantiate the profile to include radio nodes and use the provided
Shout measurement framework to perform RF measurements.
\item Instantiate the profile with both compute and radio resources to perform
a combination of the above two activities.
\end{inparaenum}

Instantiating a POWDER profile requires:
\begin{inparaenum}[(a)]
\item A user account on the POWDER platform.
\item An associated POWDER project. 
\item (Optional) For projects that require access to radio resources,
e.g., options~2 and~3 above, the project leads need to specifically request such access
by sending email to powder-support@powderwireless.net.  
\end{inparaenum}

To simply the process for TPC evaluation of our work we have created
a ``cnert21reproduce'' project, which the TPC members evaluating the reproducability
of our work can join. Specifically, for this project access to radio resources
will already be enabled. (For completeness we also provide the slightly more involved
set of steps that a ``general user'' would have to follow below). 
Simplified steps for CNERT TPC:
% to evaluate the reproducibility of our work:
\begin{enumerate}
\item Point your browser at: https://www.powderwireless.net and click on ``Sign Up'' button.
\item Sign up for an account: Provide personal information on the left (be sure to include
an ssh public key). Select the ``Join Existing Project'' option and select
``cnert21reproduce'' as the project name. Click ``Submit Request''. (Note: If you  already
have an account on POWDER, you can simply log in, click on your login ID at the
top right and select ``Start/Join Project'' from the drop-down menu.)
\item At this point you will have to wait for POWDER administrators to approve your account
and project join request. Once your account and project join request is approved (you will
receive email to that effect) you can return to
https://www.powderwireless.net to log in.
\item To instantiate the ``RF propagation validation'' profile: Point your your browser to
profile URL and follow the four step process to create an experiment: 
%

https://www.powderwireless.net/p/PowderProfiles/rf-propagation
\item Follow the instructions in the profile to perform the experiment.
\end{enumerate}

General instructions to instantiate RF propagation validation profile:
\begin{enumerate}
\item Point your browser at: https://www.powderwireless.net and click on ``Sign Up'' button.
\item Sign up for an account: Provide personal information on the left (be sure to include
an ssh public key). Select the ``Start New Project'' option and provide the requested
information. Click ``Submit Request''. (Note: If you  already
have an account on POWDER, you can simply log in, click on your login ID at the
top right and select ``Start/Join Project'' from the drop-down menu.)
\item At this point you will have to wait for POWDER administrators to approve your account
and project creation request. As explained above, if you plan to make use of POWDER radio resources,
you should send email to powder-support@powderwireless.net to make your request. 
Once your account and project is approved (you will
receive email to that effect) you can return to
https://www.powderwireless.net to log in.
\item To instantiate the ``RF propagation validation'' profile: Point your your browser to
profile URL and follow the four step process to create an experiment: 
%

https://www.powderwireless.net/p/PowderProfiles/rf-propagation
\item Follow the instructions in the profile to perform the experiment.
\end{enumerate}


%\bibliographystyle{bibtex/IEEEtran}
%\bibliography{bibtex/IEEEabrv,bibtex/bibliography}

\end{document}

