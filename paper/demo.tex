\documentclass[10pt]{article}
\usepackage[empty]{fullpage}
\usepackage{mathpazo}
\usepackage{microtype}
\usepackage{paralist}
%\usepackage{hyperref}

\begin{document}

%%%%%%%%%%%%%%%%%%%%%%%%%%%%%%%%%%%%%%%%%%%%%%%%%%%%%%%%%%%%%%%%%%%%%%%%%%%%%%%

\begin{center}
  \Large\bfseries
  Demo Proposal: Towards using the POWDER platform for RF propagation validation 
\end{center}

POWDER (the Platform for Open Wireless Data-driven Experimental Research) is a
facility for experimenting on the future of wireless networking in a city-scale ``living 
laboratory''. Powder is run by the University of Utah in partnership with Salt Lake City
and the Utah Education and Telehealth Network~\cite{b1,powderweb}.

POWDER is an end-to-end platform for research on mobile wireless networks. It provides radios 
that are programmable down to the waveform, attached to a network that can be configured by 
the user, connected to a wide variety of compute, storage, and cloud resources. Researchers can
use this platform to build their own wireless networks, using existing protocols or technologies 
(such as 4G and MIMO), up-and coming ones (such as 5G and massive MIMO), or new ones 
that they invent and build from the ground up.
POWDER includes a variety of terrain types, building sizes, and densities, making for a dynamic 
environment in which to perform real experiments.
% Within the area, there will be dozens of fixed 
%stations and a hundred couriers, such as buses and utility vehicles, carrying mobile devices 
%around the city. 

We have packaged our work into a POWDER profile that enables others to
replicate our results and to serve as a starting point for related research
efforts~\cite{paper_profile}. POWDER profiles are essentially parameterized
``recipes'' that specifies the hardware and software resources required
for the experiment in question. This ``recipe'' is used by the POWDER
control framework to allocate and configure the resources in question. 
POWDER profiles also contain detailed descriptions of what to
do once the profile has been instantiated.

In the case of the RF propagation validation profile a number of options
are possible:
\begin{enumerate}
\item Instantiate the profile with only compute resources and use the provided
SPLAT! software and POWDER related configuration information to perform 
RF propagation predictions.
\item Instantiate the profile to include radio nodes and use the provided
Shout measurement framework to perform RF measurements.
\item Instantiate the profile with both compute and radio resources. Perform
a combination of the above two activities.
\end{enumerate}

For the purpose of our demo we will:
\begin{itemize}
\item Demonstrate how to use the profile and instantiate it into an experiment using step~3 
above; i.e., our combined option. This entails reserving a set of radio nodes and a
frequency range so that we can conduct our experiment. 
\item Demonstrate how to perform RF measurements with Shout.
\item Demonstrate how to run SPLAT!.
\item Demonstrate how to produce a joint scatter plot of the two data sets (Shout and SPLAT!). 
\end{itemize}

%Please look at our reproducibility page for more information on how to use our POWDER profile. 

\bibliographystyle{bibtex/IEEEtran}
\bibliography{bibtex/IEEEabrv,bibtex/bibliography}

\end{document}

